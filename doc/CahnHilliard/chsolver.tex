\documentclass[10pt,a4paper]{article}

\usepackage[USenglish]{babel}
\usepackage[T1]{fontenc}
\usepackage[ansinew]{inputenc}
\usepackage{listings}
%\usepackage{lmodern} 
\usepackage{graphicx} %%For loading graphic files
%\usepackage{subfig} %%Subfigures inside a figure
%\usepackage{pst-all} %%PSTricks - not useable with pdfLaTeX
\usepackage{amsmath,amssymb,amsfonts,latexsym,cancel,empheq}
\usepackage{pdfpages}
\usepackage{amsthm}
\usepackage{pifont}

%% Other Packages %%%%%%%%%%%%%%%%%%%%%%%%%%%%%%%%%%%%%%%%%%%
\usepackage{a4wide} %%Smaller margins = more text per page.
\usepackage{fancyhdr} %%Fancy headings
%\usepackage{longtable} %%For tables, that exceed one page
\usepackage[small]{caption} 
\usepackage{subfigure} 			    % subfiguras
%\usepackage{color}              %para escribir con colores
\usepackage{float}              %elementos flotantes: tablas y figuras
\usepackage{appendix}
\usepackage{chngcntr}
\usepackage{etoolbox}
\usepackage{lipsum}
\usepackage{pgf,tikz}
\usetikzlibrary{shapes.misc}
 \usepackage{xcolor}
 \parindent = 0mm
 \usetikzlibrary{datavisualization}
\usetikzlibrary{matrix,calc,intersections,through,backgrounds,decorations.pathmorphing,arrows}
%\usepackage[colorlinks=true,linkcolor=black]{hyperref}
%\usepackage{tkz-fct}
\tikzset{
    line/.style={
        very thick,
        black,
        to path={% works only with "to" and not "--"
            -- (\tikztotarget) node[at start,point] {} node [at end,point] {} \tikztonodes
        }
    },
    point/.style={
        thick,
        draw=red,
        cross out,
        inner sep=0pt,
        minimum width=4pt,
        minimum height=4pt,
    },
}
 
%%%%%%%%%%%%%%%%%%%%%%%%%%%%%%%%%%%%%%%%%%%%%%%%%%%%%%%%%%%%%
%% Cabeceras y pies de p�gina
%%%%%%%%%%%%%%%%%%%%%%%%%%%%%%%%%%%%%%%%%%%%%%%%%%%%%%%%%%%%%
\lhead{DMAIA}
%\chead[y1]{y2}
\rhead{Discontinuous Galerkin Spectral Element Methods}
\renewcommand{\headrulewidth}{0.4pt}
%
%% aqui definimos el pie de pagina de las paginas pares e impares.
%\lfoot[a1]{b2}
%\cfoot[c1]{d2}
%\rfoot[e1]{f2}
%\renewcommand{\footrulewidth}{0.5pt}
\AtBeginEnvironment{subappendices}{
\chapter*{Appendix}
\addcontentsline{toc}{chapter}{Appendices}
\counterwithin{figure}{section}
\counterwithin{table}{section}
\counterwithin{equation}{section}}
\makeatletter

\makeatother
\usepackage{appendix}           %para meter ap�ndices con letras
\usepackage{dsfont}             %para poner los N,R etc. de conjuntos con $ \mathds{R} $
\usepackage{amssymb}            % para los N, R con \mathbb{R}
\usepackage{mathabx}
\usepackage{stmaryrd}

\usepackage[justification=justified]{caption} %para centrar leyendas de graficos
\usepackage{longtable}
\usepackage[colorlinks=true,linkcolor=black]{hyperref}
\usepackage[labelfont=bf]{caption}
\usepackage{pdfpages}
\definecolor{gray}{rgb}{0.75, 0.75, 0.75}
%\pagestyle{fancy}
%%%%%%%%%%%%%%%%%%%%%%%%%%%%%%%%%%%%%%%%%%%%%%%%%%%%%%%%%%%%%
%% DOCUMENT
%%%%%%%%%%%%%%%%%%%%%%%%%%%%%%%%%%%%%%%%%%%%%%%%%%%%%%%%%%%%%
\pagestyle{plain} %Now display headings: 1headings / fancy / ...
 \fancyhf{}
\pagestyle{fancy}
\fancypagestyle{plain}{}
\fancyhead[R]{}
\fancyhead[L]{Departamento de matem\'atica aplicada a la ingenier\'ia aeroespacial}
%\fancyfoot[L]{}
\fancyfoot[R]{\thepage}
%\renewcommand{\footrulewidth}{1pt}
\renewcommand{\headrulewidth}{1pt}
\makeglossary
\title{\textbf{Discontinuous Galerkin Spectral Element Method for one-dimensional conservation laws}}
%\author{Autor}
%%%%%%%%%%%%%%%%%%%%%%%%%%%%%%%%%%%%%%%%%%%%%%%%%%%%%%%%%%%%%%%%%%%%%%%%%%%%%%%%%%%%%%%%%%%%%%%%%%%%%%%%%%%%%%%%%%%%%%%%%%%%%%%%%%%%%%%%%%%%%%%%%%%%%%
\begin{document}
\tableofcontents
\section{Cahn-Hilliard equation}

\begin{align}
\frac{\partial c}{\partial t}&=\nabla\cdot M[\nabla \mu]\\
\mu &= \frac{\partial f^c(c)}{\partial c} - \kappa \nabla^2 c \\
f^c(c)&=\rho_s(c-c_\alpha)^2(c-c_\beta)^2
\end{align}

We consider the dimensionless version, where we define:

\begin{equation}
\nabla = L_{ref}\tilde{\nabla},~~ \mu = \rho_s\tilde{\mu},~~f^c = \rho_s\tilde{f}^c,~~w=\sqrt{\frac{\kappa}{\rho_s L_{ref}^2}},~~ \sigma = \sqrt{\frac{\kappa\rho_s}{L_{ref}^2}},~~t_c = \frac{L_{ref}^2}{M}
\end{equation}

Such that:

\begin{align}
\tilde{f}^c &= (c-c_\alpha)^2(c-c_\beta)^2\\
\tilde{\mu} &= \frac{\partial \tilde{f}^c}{\partial c} - w^2 \tilde{\nabla}^2 c\\
\frac{\partial c}{\partial \tau} &= \tilde{\nabla}^2 \tilde{\mu}
\end{align}

The input parameters are:

\begin{equation}
L_{ref}, w, \sigma, M
\end{equation}

\section{Discretization of fourth-order derivatives (a.k.a. bi-Laplacian)}

We consider the equation

\begin{equation}
f = \nabla^4 u
\end{equation}

which can be solved using the following primal form:

\begin{equation}
\begin{split}
\int_{\Omega}fvdx =&~~~ \int_{\Omega}\nabla^2 u \nabla^2 vdx - \int_{\mathcal{E}}\llbracket\nabla v \rrbracket \{\!\{\nabla^2 u\}\!\}ds - \int_{\mathcal{E}}\llbracket \nabla u\rrbracket \{\!\{\nabla^2 v\}\!\}ds\\
&+\int_{\mathcal{E}}\{\!\{\nabla\nabla^2 u\}\!\}\llbracket v\rrbracket ds + \int_{\mathcal{E}}\{\!\{\nabla \nabla^2 v\}\!\}\llbracket u\rrbracket ds \\
&+\frac{\sigma_0}{h^3}\int_{\mathcal{E}}\llbracket u\rrbracket \llbracket v\rrbracket ds
\end{split}
\end{equation}


\subsection{One-dimensional version}

\begin{equation}
\begin{split}
&\boldsymbol{D_2}^T \boldsymbol{M}\boldsymbol{D_2}\underline{u}^k +T_{u3} + T_{v3} - T_{u2} - T_{v2}+T_{\sigma} \\
&
\end{split}
\end{equation}

\begin{equation}
\begin{split}
T_{u3} &= \frac{1}{2}\biggl[\boldsymbol{L}_{1}^1 \boldsymbol{D}_3\boldsymbol{\underline{u}}^k+
\boldsymbol{L}_{1}^{-1} \boldsymbol{D}_3\boldsymbol{\underline{u}}^{k+1} -  
\boldsymbol{L}_{-1}^{-1} \boldsymbol{D}_3\boldsymbol{\underline{u}}^k - 
\boldsymbol{L}_{-1}^{1} \boldsymbol{D}_3\boldsymbol{\underline{u}}^{k-1}\biggr] \\
T_{v3} &= \frac{1}{2}\biggl[\boldsymbol{D}^{T}_3\boldsymbol{L}_{1}^1 \boldsymbol{\underline{u}}^k-
\boldsymbol{D}^{T}_3\boldsymbol{L}_{1}^{-1} \boldsymbol{\underline{u}}^{k+1} - \boldsymbol{D}^{T}_3\boldsymbol{L}_{-1}^{-1} \boldsymbol{\underline{u}}^k + \boldsymbol{D}^{T}_3\boldsymbol{L}_{-1}^{1} \boldsymbol{\underline{u}}^{k-1}\biggr] \\
T_{u2} &= \frac{1}{2}\biggl[\boldsymbol{D}^T \boldsymbol{L}_{1}^1 \boldsymbol{D}_2 \underline{u}^k + \boldsymbol{D}^T \boldsymbol{L}_{1}^{-1} \boldsymbol{D}_2 \underline{u}^{k+1} -\boldsymbol{D}^T \boldsymbol{L}_{-1}^1 \boldsymbol{D}_2 \underline{u}^{k-1} - \boldsymbol{D}^T \boldsymbol{L}_{-1}^{-1} \boldsymbol{D}_2 \underline{u}^{k}  \biggr]\\
T_{v2} &= \frac{1}{2}\biggl[\boldsymbol{D}_2^T \boldsymbol{L}_{1}^1  \boldsymbol{D}^T\underline{u}^k - \boldsymbol{D}_2^T \boldsymbol{L}_{1}^{-1}  \boldsymbol{D}^T \underline{u}^{k+1} +
 \boldsymbol{D}_2^T \boldsymbol{L}_{-1}^{1}  \boldsymbol{D}^T \underline{u}^{k-1} -
  \boldsymbol{D}_2^T \boldsymbol{L}_{-1}^{-1}  \boldsymbol{D}^T\underline{u}^{k}  \biggr]\\
T_\sigma &= \frac{\sigma_0}{h^3}\biggl[ \boldsymbol{L}_{1}^1 \underline{u}^{k} +  \boldsymbol{L}_{-1}^{-1} \underline{u}^{k} - \boldsymbol{L}_{1}^{-1} \underline{u}^{k+1}- \boldsymbol{L}_{-1}^{1} \underline{u}^{k-1}\biggr]
\end{split}
\end{equation}


Left, central, and right matrices are:

\begin{equation}
\begin{split}
\boldsymbol{L} & = \frac{1}{2}\bigl[-\boldsymbol{L}_{-1}^{1} \boldsymbol{D}_3 + \boldsymbol{D}^{T}_3\boldsymbol{L}_{-1}^{1} +\boldsymbol{D}^T \boldsymbol{L}_{-1}^1 \boldsymbol{D}_2 - \boldsymbol{D}_2^T \boldsymbol{L}_{-1}^{1}  \boldsymbol{D}\bigr] -\frac{\sigma_0}{h^3}\boldsymbol{L}_{-1}^{1}\\ 
\boldsymbol{C} & = \boldsymbol{D_2}^T \boldsymbol{M}\boldsymbol{D_2} + \text{sym}\biggl[\bigl(\boldsymbol{L}_{1}^1 -\boldsymbol{L}_{-1}^{-1}\bigr) \boldsymbol{D}_3-\boldsymbol{D}^T \bigl(\boldsymbol{L}_{1}^1 -\boldsymbol{L}_{-1}^{-1}\bigr)\boldsymbol{D}_2\biggr] + \frac{\sigma_0}{h^3}\bigl(\boldsymbol{L}_{1}^1 +\boldsymbol{L}_{-1}^{-1}\bigr)\\
\boldsymbol{R} & = \frac{1}{2}\bigl[\boldsymbol{L}_{1}^{-1} \boldsymbol{D}_3 - \boldsymbol{D}^{T}_3\boldsymbol{L}_{1}^{-1} -\boldsymbol{D}^T \boldsymbol{L}_{1}^{-1} \boldsymbol{D}_2 + \boldsymbol{D}_2^T \boldsymbol{L}_{1}^{-1}  \boldsymbol{D}^T\bigr] -\frac{\sigma_0}{h^3}\boldsymbol{L}_{1}^{-1}\\
\end{split}
\end{equation}


\end{document}