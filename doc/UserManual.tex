\documentclass[a4paper,10pt]{report}
%\usepackage[spanish]{babel}
\usepackage[utf8]{inputenc}
\usepackage{graphicx}
\usepackage{float} 
\usepackage{multirow}
\usepackage{vmargin}
\usepackage{longtable}
\usepackage{amsmath}
\usepackage{amsfonts}
\usepackage{subfigure}
\usepackage{afterpage}
\usepackage[calcwidth]{titlesec}
\usepackage{verbatim}
\usepackage[hidelinks]{hyperref}
\usepackage{multicol}
\usepackage{pfnote}
\usepackage{fnpos}
\usepackage{color}
\usepackage{xcolor}
\usepackage{listings}
\usepackage{algorithm,algpseudocode}
\usepackage{epstopdf}
\usepackage{tcolorbox}
\usepackage{amssymb}
\setmarginsrb{2cm}{2cm}{2cm}{2cm}{0cm}{0cm}{0cm}{0.5cm}%{left}{top}{right}{bottom}{headhgt}{}
%\numberwithin{equation}{section}
%Bibliography style:
\bibliographystyle{elsarticle-num}
%\biboptions{sort&compress}
%opening

\definecolor{mygreen}{RGB}{28,172,0} % color values Red, Green, Blue
\definecolor{mylilas}{RGB}{170,55,241}

%% Some math definitions:
\newcommand\norm[1]{\left\lVert#1\right\rVert}
\def\Res{\pmb{\mathfrak{R}}}


\title{\textbf{HORSES3D} \\ A \textbf{H}igh-\textbf{Or}der (DG) \textbf{S}pectral \textbf{E}lement \textbf{S}olver \\ \textbf{User Manual}}
\author{Andrés Rueda \\ Many others (future?)}

\begin{document}

\lstset{language=C++}

\maketitle

\tableofcontents

\chapter{Compiling the code}

\begin{itemize}
\item Clone the git repository or copy the source code into a desired folder.

\item Go to the folder Solver.

\item Run configure script.
\begin{lstlisting}[language=bash]
	$ ./configure
\end{lstlisting}
\item Install using the Makefile:
\begin{lstlisting}[language=bash]
	$ make all <<Options>>
\end{lstlisting}
with the desired options (bold are default):

\begin{itemize}
\item MODE=DEBUG/\textbf{RELEASE}
\item COMPILER=ifort/\textbf{gfortran}
\item COMM=PARALLEL/\textbf{SEQUENTIAL}
\item PLATFORM=MACOSX/\textbf{LINUX}
\item ENABLE\_THREADS=NO/\textbf{YES}
\end{itemize}

For example:
\begin{lstlisting}[language=bash]
	$ make all COMPILER=ifort COMM=PARALLEL
\end{lstlisting}

\item If you use \textit{environment modules}, it is advised to use the HORSES3D module file:\\
\begin{lstlisting}[language=bash]
	$ export MODULEPATH=$HORSES_DIR/utils/modulefile:$MODULEPATH
\end{lstlisting}
where \$HORSES\_DIR is the installation directory.
\end{itemize}



\chapter{Input and Output Files}

DONT USE TABS!

\section{Input Files}
\begin{itemize}
\item Control file (*.control)
\item Mesh file (*.mesh / *.h5)
\item Polynomial order file (*.omesh)
\item Problem File (ProblemFile.f90)
\end{itemize}

\section{Output Files}
\begin{itemize}
\item Solution file (*.hsol)
\item Horses mesh file (*.hmesh)
\item Boundary information (*.bmesh)
\item Partition file (*.pmesh)
\item Polynomial order file (*.omesh)
\item Monitor files (*.volume / *.surface / *.residuals)
\end{itemize}

\chapter{Running a Simulation}

\section{Control File (*.control) - Overview}
The control file is the main file for running a simulation. A list of all the mandatory keywords for running a simulation and some basic optional keywords is presented in Table \ref{tab:runningkey}. The specific keywords are listed in the other chapters.

\begin{longtable}{|p{4cm}|p{10cm}|p{2.2cm}|}
\caption{General keywords for running a case.} \label{tab:runningkey} \\
\hline
\multicolumn{1}{|c|}{\textbf{Keyword}} & \multicolumn{1}{c|}{\textbf{Description}} & \multicolumn{1}{c|}{\textbf{Default value}} \\ \hline
\endfirsthead

\caption{General keywords for running a case - continued.} \\
\hline
\multicolumn{1}{|c|}{\textbf{Keyword}} & \multicolumn{1}{c|}{\textbf{Description}} & \multicolumn{1}{c|}{\textbf{Default value}} \\ \hline
\endhead

solution file name   & \textit{CHARACTER}: Path and name of the output file. The name of this file is used for naming other output files. & \textbf{Mandatory\ keyword} \\ \hline

simulation type        & \textit{CHARACTER}: Specifies if NSLITE3D must perform a 'steady-state' or a 'time-accurate' simulation. &  'steady-state'\\ \hline

time integration & \textit{CHARACTER}: Can be 'implicit', 'explicit', or 'FAS'. The latter uses the Full Algebraic Storage (FAS) multigrid scheme, which can have implicit or explicit smoothers. & 'explicit' \\ \hline

polynomial order   & \textit{INTEGER}: Polynomial order to be assigned uniformly to all the elements of the mesh. If the keyword \textit{polynomial order file} is specified, the value of this keyword is overridden. & --* \\ \hline

polynomial order i \

polynomial order j \

polynomial order k & \textit{INTEGER}: Polynomial order in the i, j, or k component for all the elements in the domain. If used, the three directions must be declared explicitly, unless you are using a polynomial order file. If the keyword \textit{polynomial order file} is specified, the value of this keyword is overridden. & --* \\ \hline

polynomial order file  & \textit{CHARACTER}: Path to a file containing the polynomial order of each element in the domain. & --* \\ \hline

restart 			& \textit{LOGICAL}: If .TRUE., initial conditions of simulation will be read from restart file specified using the keyword \textit{restart file name}. & \textbf{Mandatory keyword} \\ \hline

cfl & \textit{REAL}: A constant related with the \textbf{convective} Courant-Friedrichs-Lewy (CFL) condition that the program will use to compute the time step size. & --** \\ \hline

dcfl & \textit{REAL}: A constant related with the \textbf{diffusive} Courant-Friedrichs-Lewy (DCFL) condition that the program will use to compute the time step size. & --** \\ \hline

dt  & \textit{REAL}: Constant time step size.  & --** \\ \hline

final time  & \textit{REAL}: This keyword is mandatory for time-accurate solvers & -- \\ \hline

mesh file name & \textit{CHARACTER}: Name of the mesh file. The currently supported formats are \textit{.mesh} (SpecMesh file format) and \textit{.h5} (HOPR hdf5 file format). & \textbf{Mandatory\ keyword} \\ \hline

mesh inner curves & \textit{LOGICAL}: Specifies if the mesh reader must suppose that the inner surfaces (faces connecting the elements of the mesh) are curved. This input variable only affects the hdf5 mesh reader. & .TRUE. \\ \hline

number of time steps & \textit{INTEGER}: \textit{Maximum} number of time steps that the program will compute.  & \textbf{Mandatory\ keyword} \\ \hline

output interval   & \textit{INTEGER}: In steady-state, this keyword indicates the interval of time steps to display the residuals on screen. In time-accurate simulations, this keyword indicates how often a 3D output file must be stored.  & \textbf{Mandatory\ keyword} \\ \hline

convergence tolerance & \textit{REAL}: Residual convergence tolerance for steady-state cases & \textbf{Mandatory\ keyword} \\ \hline

manufactured solution & \textit{CHARACTER}: Must have the value '2D' or '3D'. When this keyword is used, the program will add source terms for the conservative variables taken into account an exact analytic solution for each primitive variable j ($\rho$, $u$, $v$, $w$, $p$) of the form:\
 
$j = j_C(1) + j_C(2) \sin(\pi j_C(5) x) + j_C(3) \sin(\pi j_C(6) y) + j_C(4) \sin(\pi j_C(7) z) $\

Where $j_C(i)$ are constants defined in the file \textit{ManufacturedSolutions.f90}. Proper initial and boundary conditions must be imposed (see the test case). The mesh must be a unit cube.
  & -- \\ \hline

\multicolumn{3}{p{16.4cm}}{*  One of these keywords must be specified} \\

\multicolumn{3}{p{16.4cm}}{** For Euler simulations, the user must specify either the CFL number or the time-step size. For Navier-Stokes simulations, the user must specify the CFL and DCFL numbers \textbf{or} the time-step size.}

\end{longtable}

\section{Boundary conditions}

Juan's email (to be translated and adapted to the manual format):

Hola Gente,

He tenido que hacer unas modificaciones bastante importantes en las BCs. Era la única parte del código que estaba “a la antigua” y no programada a objetos. Esto hacía que no fueran muy customizables, y por ejemplo las controlábamos con el número ese que siempre vale 0.0 jajaja. Ahora cada condición de contorno tiene los parámetros que necesitas y se pueden customizar. Lo malo es que ningún control file de los que tenéis van a seguir funcionando, pero os escribo los cambios para que sepáis adaptarlos, en cualquier caso, podéis pedirme ayuda y os cuento.

Los cambios del código son:

\begin{itemize}

\item Las condiciones de contorno se definen igual que los monitores, con los $\#$define en la parte final del control file. Para definir una condicion de contorno se hace:
\begin{lstlisting}
        #define boundary name
             type = Inflow/Outflow/NoSlipWall/FreeSlipWall/Periodic/User-defined
             parametro1 = #valor
             parametro2 = #valor
        #end
\end{lstlisting}
\item Los parámetros1, … dependen de la condición de contorno que toque. Si no se especifica nada, pues está como estaba antes. Dos cambios importantes:

          · He unificado las NoSlipWall (adiabatica e isoterma) en una sola. Por defecto es adiabática.
          · En las periódicas es obligatorio ahora indicar a qué boundary se acopla (lo cual supone poco esfuerzo y reduce el tiempo de búsqueda al código)
\begin{lstlisting}
                 #define boundary name
                       type = Periodic
                      coupled boundary = nombredelboundaryalqueseacopla
                 #end
\end{lstlisting}
\item Se pueden definir más de una condición de contorno del tirón, por ejemplo si boundary1, boundary2 y boundary3 son inflows se puede hacer:
\begin{lstlisting}
                 #define boundary boundary1__boundary2__boundary3
                         type = Inflow
                 #end
\end{lstlisting}
     es decir, separado con dos guiones bajos.

\item Por pantalla, donde aparecía la info de las zones y tal, también aparece qué BC tiene y cuáles son los parámetros.

\item La BC outflowspecifyP la llamo simplemente Outflow. Más que nada por que antes había algunos ficheros de control con la BC Outflow y no existía, pero por defecto se mandaba a Inflow. Para evitar problemas, pues Outflow.

\item Los archivos de condición de contorno están en /physics/common en lugar de cada uno su archivo. Esto es por que al final son todas iguales y si se añade una nueva es más facil agregar un nuevo archivo que hacerlo individualmente en cada ecuación.

\item Los bcTypeDictionary bcValueDictionary desaparecen. Las BC están en el module physics/common/BoundaryConditions.f90 como variable global, se llama BCs y dentro aloja todas las condiciones de contorno (una por zona, y en el mismo orden de las zonas).

\end{itemize}

Creo que eso es todo, lamento si os supone mucho cambio en vuestros ficheros de control que estéis corriendo a día de hoy, y si rompo algo que no reflejen los test. Pero estos cambios eran necesarios para darle más versatilidad (por ejemplo en multifase el inflow necesita bastante customización, para definir caudales de cada fase y cosas así). Además, creo que el enfoque OOP va en la dirección del resto del código.


\chapter{Restarting a Case}

\begin{table}[htbp]
\caption{Keywords for restarting a case.}
\begin{tabular}{|l|p{10cm}|p{2.2cm}|}
\hline
\multicolumn{1}{|c|}{Keyword} & \multicolumn{1}{c|}{Description} & \multicolumn{1}{c|}{Default value} \\ \hline

restart 			& \textit{LOGICAL}: If .TRUE., initial conditions of simulation will be read from restart file specified using the keyword \textit{restart file name}. & \textbf{Mandatory keyword} \\ \hline

restart file name   & \textit{CHARACTER}: Name of the restart file to be written and, if keyword \textit{restart} = .TRUE., also name of the restart file to be read for starting the simulation. & \textbf{Mandatory\ keyword} \\ \hline

restart polorder & \textit{INTEGER}: Uniform polynomial order of the solution to restart from. This keyword is only needed when the restart solution is of a different order than the current case. & same as case's \\ \hline

restart polorder file & 
			\textit{CHARACTER}: File containing the polynomial orders of the solution to restart from. This keyword is only needed when the restart solution is of a different order than the current case.  &  same as case's\\ \hline

get discretization error of & \textit{CHARACTER}: Path to solution file. This can be used to estimate the discretization error of a solution when restarting from a higher-order solution. & -- \\ \hline

\end{tabular}
\label{tab:restartkey}
\end{table}

%%%%%%%%%%%%%%%%%%%%%%%%%%%%%%%%%%%%%%%%%%%%%%%%%%%%%%%%%%%%%%%%%%%%%%%%%

\chapter{Physics related keyword}

\section{Compressible flow}

\begin{table}[htbp]
\caption{Keywords for compressible flow (Euler / Navier-Stokes).}
\begin{tabular}{|l|p{10cm}|p{2.2cm}|}
\hline
\multicolumn{1}{|c|}{Keyword} & \multicolumn{1}{c|}{Description} & \multicolumn{1}{c|}{Default value} \\ \hline

Mach number					& \textit{REAL}:  & \textbf{Mandatory keyword} \\ \hline

Reynolds number				& \textit{REAL}:  & \textbf{Mandatory keyword}
 \\ \hline

Prandtl number				& \textit{REAL}:  & 0.72 \\ \hline

Turbulent Prandtl number	& \textit{REAL}:  & Equal to Prandtl \\ \hline

LES model					& \textit{CHARACTER(*)}: Options are: 
\begin{itemize}
\item Smagorinsky
\item None
\end{itemize}
 & None \\ \hline

Wall model					& \textit{CHARACTER(*)}:  & linear \\ \hline

\end{tabular}
\label{tab:compressibleFlowkey}
\end{table}

\section{Incompressible Navier-Stokes}

\section{Cahn-Hilliard}

\chapter{Implicit Solvers with Newton linearisation}
\section{General Keywords}
The keywords for the implicit solvers are listed in table \ref{tab:implicitkey}

%\begin{table}[htbp]
%\caption{Keywords for implicit solvers.}
%\begin{tabular}{|l|p{10cm}|p{2.2cm}|}
%\hline
%\multicolumn{1}{|c|}{Keyword} & \multicolumn{1}{c|}{Description} & \multicolumn{1}{c|}{Default value} \\ \hline

\begin{longtable}{|p{4cm}|p{10cm}|p{2.2cm}|}
\caption{Keywords for implicit solvers.} \label{tab:implicitkey} \\
\hline
\multicolumn{1}{|c|}{\textbf{Keyword}} & \multicolumn{1}{c|}{\textbf{Description}} & \multicolumn{1}{c|}{\textbf{Default value}} \\ \hline
\endfirsthead

\caption{Keywords for implicit solvers - continued.} \\
\hline
\multicolumn{1}{|c|}{\textbf{Keyword}} & \multicolumn{1}{c|}{\textbf{Description}} & \multicolumn{1}{c|}{\textbf{Default value}} \\ \hline
\endhead

\textbf{time integration} & \textit{CHARACTER}: This is the main keyword for activating the implicit solvers. The value of it should be set to 'implicit' for the BDF solvers and to 'rosenbrock' for Rosenbrock schemes. & 'explicit' \\ \hline

linear solver           & \textit{CHARACTER}: Specifies the linear solver that has to be used. Options are:\ 
				\begin{itemize}
					\item 'petsc': PETSc library Krylov-Subspace methods. Available in serial, but use with care (PETSc is not thread-safe, so OpenMP is not recommended). Only available in parallel (MPI) for preallocated Jacobians (see next section).
					\item 'pardiso': Intel MKL PARDISO. Only available in serial or with OpenMP.
					\item 'matrix-free gmres': A matrix-free version of the GMRES algorithm. Can be used without preconditioner or with a recursive GMRES preconditioner using 'preconditioner=GMRES'. Available in serial and parallel (OpenMP+MPI)
					\item 'smooth': Traditional iterative methods. One can select either 'smoother=WeightedJacobi' or 'smoother=BlockJacobi'.
					\item 'matrix-free smooth': A matrix-free version of the previous solver. Only available with 'smoother=BlockJacobi'.
				\end{itemize}
										& 'petsc'  \\ \hline



\end{longtable}

\section{Keywords for the BDF Methods}

The BDF methods implemented in HORSES3D use a Newton's method

\begin{longtable}{|p{4cm}|p{9cm}|p{3.2cm}|}
\caption{Keywords for the BDF solvers.} \label{tab:BDFkey} \\
\hline
\multicolumn{1}{|c|}{\textbf{Keyword}} & \multicolumn{1}{c|}{\textbf{Description}} & \multicolumn{1}{c|}{\textbf{Default value}} \\ \hline
\endfirsthead

\caption{Keywords for the BDF solvers - continued.} \\
\hline
\multicolumn{1}{|c|}{\textbf{Keyword}} & \multicolumn{1}{c|}{\textbf{Description}} & \multicolumn{1}{c|}{\textbf{Default value}} \\ \hline
\endhead

bdf order             & \textit{INTEGER}: If present, the solver uses a BDF solver of the specified order. BDF1 - BDF5 are available, and BDF2 - BDF5 require constant time steps. & 1 \\ \hline

jacobian by convergence & \textit{LOGICAL}: When .TRUE., the Jacobian is only computed when the convergence falls beneath a threshold (hard-coded). This improves performance.  & .FALSE. \\ \hline

compute jacobian every & \textit{INTEGER}: Forces the Jacobian to be computed in an interval of iterations that is specified. & Inf \\ \hline

%% Keywords fot the Newton's method

print newton info       & \textit{LOGICAL}: If .TRUE., the information of the Newton iterations will be displayed. &  '.FALSE.'\\ \hline
implicit adaptive dt  & \textit{LOGICAL}: Specifies if the time-step should be computed according to the convergence behavior of the Newton iterative method and the linear solver. & .FALSE. \\ \hline

newton tolerance   & \textit{REAL}: Specifies the tolerance for the Newton's method. &  $10^{-6}$ for time-accurate simulations, or $MaxResidual \times a$ for steady-state simulations, where $a$ is the keyword \textit{newton factor} \\ \hline

newton max iter    & \textit{INTEGER}: Maximum number of Newton iterations for BDF solver.  & 30 \\ \hline

linsolver max iter	& \textit{INTEGER}: Maximum number of iterations to be taken by the linear solver. This keyword only affects iterative linear solvers.		    & 500 \\ \hline

newton factor 	& \textit{REAL}: In simulations that are not time-accurate, the tolerance of the Newton's method is a function of the residual: $MaxResidual \times a$, where $a$ is the specified value.	& $10^{-3}$	\\ \hline

%% Keywords for the linear solver tolerance

linsolver tol factor 	& \textit{REAL}: The linear solver tolerance is a function of the absolute error of the Newton's method: $tol=\norm{e}_{\infty}*a^i$, where $e$ is the absolute error of the Newton's method, $i$ is the Newton iteration number, and $a$ is the specified value.	& $0.5$	\\ \hline

newton first norm  & \textit{REAL}: Specifies an assumed infinity norm of the absolute error of the Newton's method at the iteration $0$ of the time step $1$. 
This can change the behavior of the first Newton iterative method because of the dependency of the linear system tolerance on the absolute error of the Newton's method (see keyword \textit{linsolver tol factor}).
				   & $0.2$ \\ \hline 







\end{longtable}

\section{Keywords for the Rosenbrock-Type Implicit Runge-Kutta Methods}

\begin{longtable}{|p{4cm}|p{10cm}|p{2.2cm}|}
\caption{Keywords for the Rosenbrock schemes.} \label{tab:Rosenbrockkey} \\
\hline
\multicolumn{1}{|c|}{\textbf{Keyword}} & \multicolumn{1}{c|}{\textbf{Description}} & \multicolumn{1}{c|}{\textbf{Default value}} \\ \hline
\endfirsthead

\caption{Keywords for the Rosenbrock schemes - continued.} \\
\hline
\multicolumn{1}{|c|}{\textbf{Keyword}} & \multicolumn{1}{c|}{\textbf{Description}} & \multicolumn{1}{c|}{\textbf{Default value}} \\ \hline
\endhead

rosenbrock scheme	& \textit{CHARACTER}: Rosenbrock scheme to be used. Currently, only the \textit{RO6-6} is implemented.		    & -- \\ \hline

\end{longtable}

\section{Jacobian Specifications}
The Jacobian must be defined using a block of the form:
\begin{lstlisting}
#define Jacobian
   type = 2
   print info = .TRUE.
   preallocate = .TRUE.
#end
\end{lstlisting}

\begin{longtable}{|p{4cm}|p{10cm}|p{2.2cm}|}
\caption{Keywords for Jacobian definition block.} \label{tab:Jacobiankey} \\
\hline
\multicolumn{1}{|c|}{\textbf{Keyword}} & \multicolumn{1}{c|}{\textbf{Description}} & \multicolumn{1}{c|}{\textbf{Default value}} \\ \hline
\endfirsthead

\caption{Keywords for Jacobian definition block - continued.} \\
\hline
\multicolumn{1}{|c|}{\textbf{Keyword}} & \multicolumn{1}{c|}{\textbf{Description}} & \multicolumn{1}{c|}{\textbf{Default value}} \\ \hline
\endhead


type             & \textit{INTEGER}: Specifies the type of Jacobian matrix to be computed. Options are:\
				\begin{enumerate}
					\item Numerical Jacobian: Uses a coloring algorithm and a finite difference method to compute the DG Jacobian matrix (only available with shared memory parallelization).
					\item Analytical Jacobian: Available with shared (OpenMP) or distributed (MPI) memory parallelization for advective and/or diffusive nonlinear conservation laws, \textbf{BUT} only for the standard DGSEM (no split-form). 
					\end{enumerate}
										& \textbf{Mandatory Keyword} \\ \hline

print info      & \textit{LOGICAL}: Specifies the verbosity of the Jacobian subroutines  & .TRUE. \\ \hline

preallocate     & \textit{LOGICAL}: Specifies if the Jacobian must be allocated in preprocessing (.TRUE. - only available for advective/diffusive nonlinear conservation laws) or every time it is computed (.FALSE.)  & .FALSE. \\ \hline

\end{longtable}

%%%%%%%%%%%%%%%%%%%%%%%%%%%%%%%%%%%%%%%%%%%%%%%%%%%%%%%%%%%%%%%%%%%%%%%%%%

\chapter{Explicit Solvers}

Explicit time integration schemes available in HORSES3D.
The main keywords to use it are shown in Table \ref{tab:explicitKey}.

\begin{longtable}{|p{4cm}|p{10cm}|p{2.2cm}|}
\caption{Keywords for the multigrid solver.} \label{tab:explicitKey} \\
\hline
\multicolumn{1}{|c|}{\textbf{Keyword}} & \multicolumn{1}{c|}{\textbf{Description}} & \multicolumn{1}{c|}{\textbf{Default value}} \\ \hline
\endfirsthead

\caption{Keywords for the multigrid solver - continued.} \\
\hline
\multicolumn{1}{|c|}{\textbf{Keyword}} & \multicolumn{1}{c|}{\textbf{Description}} & \multicolumn{1}{c|}{\textbf{Default value}} \\ \hline
\endhead

\textbf{time integration} & \textit{CHARACTER}: This is the main keyword to activate the multigrid solvers. The value of it should be set to 'FAS' for the Full Approximation Scheme (FAS) nonlinear multigrid  solvers and to 'AnisFAS' for anisotropic FAS schemes. & 'explicit' \\ \hline

\textbf{simulation type} & \textit{CHARACTER}: Specifies if HORSES3D must perform a ’steady-state’ or a ’time-accurate’. If 'time-accurate' the solver switches to BDF integration and uses FAS as a pseudo problem solver. Compatible only with 'FAS'. & 'steady-state' \\ \hline

explicit method & \textit{CHARACTER}: Select desired Runge-Kutta solver. Options are: 'Euler', 'RK3', 'RK5' and 'RKOpt'. & RK3 \\ \hline

rk order & \textit{INTEGER}: Order of Runge-Kutta method optimized for steady-state solver ('RKOpt'). Possible orders are from 2 to 7. & 2 \\ \hline

\end{longtable}

%%%%%%%%%%%%%%%%%%%%%%%%%%%%%%%%%%%%%%%%%%%%%%%%%%%%%%%%%%%%%%%%%%%%%%%%%


\chapter{Nonlinear $p$-Multigrid solver (FAS)}

The code has an implementation of the Full Approximation Scheme (FAS) nonlinear $p$-multigrid method. The main keywords to use it are shown in Table \ref{tab:multigridKey}.

\begin{longtable}{|p{4cm}|p{10cm}|p{2.2cm}|}
\caption{Keywords for the multigrid solver.} \label{tab:multigridKey} \\
\hline
\multicolumn{1}{|c|}{\textbf{Keyword}} & \multicolumn{1}{c|}{\textbf{Description}} & \multicolumn{1}{c|}{\textbf{Default value}} \\ \hline
\endfirsthead

\caption{Keywords for the multigrid solver - continued.} \\
\hline
\multicolumn{1}{|c|}{\textbf{Keyword}} & \multicolumn{1}{c|}{\textbf{Description}} & \multicolumn{1}{c|}{\textbf{Default value}} \\ \hline
\endhead

\textbf{time integration} & \textit{CHARACTER}: This is the main keyword to activate the multigrid solvers. The value of it should be set to 'FAS' for the Full Approximation Scheme (FAS) nonlinear multigrid  solvers and to 'AnisFAS' for anisotropic FAS schemes. & 'explicit' \\ \hline

\textbf{simulation type} & \textit{CHARACTER}: Specifies if HORSES3D must perform a ’steady-state’ or a ’time-accurate’. If 'time-accurate' the solver switches to BDF integration (the exact method can be set using 'bdf order' option) and uses FAS as a local steady-state problem solver. Compatible only with 'FAS'. & 'steady-state' \\ \hline

multigrid levels & \textit{INTEGER}: Number of multigrid levels for the computations. & \textbf{Mandatory keyword} \\ \hline

delta n          & \textit{INTEGER}: Interval of reduction of polynomial order for creating coarser multigrid levels.& 1 \\ \hline

multigrid output & \textit{LOGICAL}: If .TRUE., the residuals at the different multigrid levels will be displayed. & .FALSE. \\ \hline
   
mg sweeps    & \textit{INTEGER}: Number of smoothing sweeps to be taken. & 1* \\ \hline
    
mg sweeps pre    & \textit{INTEGER}: Number of pre-smoothing sweeps to be taken. & 1* \\ \hline
    
mg sweeps post    & \textit{INTEGER}: Number of post-smoothing sweeps to be taken. & 1* \\ \hline    

mg sweeps coarsest   & \textit{INTEGER}: Number of pre- and post-smoothing sweeps to be taken on the coarsest multigrid level. & Average between pre-sweeps and post-sweeps \\ \hline

mg sweeps exact & \textit{INTEGER(:)}: Alternative to 'mg sweeps'. Defines exact number of pre- and post- smoothing sweeps to be taken on each level. Index of the array indicates the MG level for the sweeps to be performed, e.g. [1,4] performs 1 pre-sweep and 1 post-sweep on level 1 and 4 pre-\/post-sweeps on level 2. & 1* \\ \hline

mg sweeps pre exact & \textit{INTEGER(:)}: Alternative to 'mg sweeps pre'. Defines exact number of pre-smoothing sweeps to be taken on each level. Index of the array indicates the MG level for the sweeps to be performed, e.g. [1,4] performs 1 pre-sweep on level 1 and 4 pre-sweeps on level 2. & 1* \\ \hline

mg sweeps post exact & \textit{INTEGER(:)}: Alternative to 'mg sweeps post'. Defines exact number of post-smoothing sweeps to be taken on each level. Index of the array indicates the MG level for the sweeps to be performed, e.g. [1,4] performs 1 post-sweep on level 1 and 4 post-sweeps on level 2. & 1* \\ \hline

mg smoother     & \textit{CHARACTER}: The smoothing technique to be used. The keywords and possible explicit smoothers are the same as the 'explicit method' in \ref{tab:explicitKey}. For the semi-implicit residual relaxation use 'BIRK5'. & RK3 \\ \hline

%% FMG keywords

fasfmg residual & \textit{REAL}: When this keyword is used, the code uses a full multigrid (FMG) method to obtain an initial condition for the simulation. 
The initial condition has the specified residual.	& --	\\ \hline

fasfmg save solutions & \textit{LOGICAL}: Save the solutions that are obtained at the different FMG levels. 
Only usable when \textit{fasfmg residual} is used. 	& .FALSE.	\\ \hline

%% Smoothing tuning keywords

postsmooth option    & \textit{CHARACTER}: When this keyword is used, the code performs extra post-smoothing sweeps, so that the final residual after completing the post-smoothing is lower than the residual achieved by the pre-smoothing. The options are:\

\begin{itemize}
\item \textit{f-cycle}: Do the extra post-smoothing with an FMG cycle.
\item \textit{smooth}: Do normal smoothing.
\end{itemize}  & -- \\ \hline

smooth fine & \textit{REAL}: Extra pre-smoothing is performed on a multigrid level of order $P$, until a residual is obtained $\norm{\tilde{\Res}^{P}}_{\infty} < \eta \norm{\tilde{\Res}^{N}}_{\infty}$, where $N$ is the polynomial order of the next (coarsest) grid, and $\eta$ is the specified value.	& --	\\ \hline

max mg sweeps & \textit{INTEGER}: Maximum number of smoothing sweeps to be performed. This only makes sense if one uses the keywords \textit{postsmooth option} and/or \textit{smooth fine}.	& 10000	\\ \hline

mg initialization & \textit{LOGICAL}: Sets the initial explicit residual smoothing with RK3 and local time stepping. & .FALSE.	\\ \hline

initial residual & \textit{REAL}: Threshold for the $\norm{\tilde{\Res}^{P}}_{\infty}$ after which solver switches from the 'mg initialization' settings to user specified. & 1.0	\\ \hline

initial cfl & \textit{REAL}: CFL and DCFL number for initial residual smoothing. & 0.1	\\ \hline

\multicolumn{3}{p{16.4cm}}{*  The user must specify \textit{mg sweeps pre} \textbf{and} \textit{mg sweeps post}, or \textit{mg sweeps}.} \\

\end{longtable}

%%%%%%%%%%%%%%%%%%%%%%%%%%%%%%%%%%%%%%%%%%%%%%%%%%%%%%%%%%%%%%%%%%%%%%%%%
\chapter{p-Adaptation Methods}
The p-adaptation methods are used when the p-adaptation region is specified in the control file:\\

\begin{lstlisting}
#define p-adaptation
   Truncation error type = isolated
   truncation error      = 1.d-2
   Nmax                  = [10,10,10]
   Nmin                  = [2 ,2 ,2 ]
   Conforming boundaries = [InnerCylinder,sphere]
   order across faces    = N*2/3   
   increasing            = .FALSE.
   write error files     = .FALSE.
   adjust nz             = .FALSE.
   mode                  = time
   interval              = 1.d0   
   restart files         = .TRUE.
   max N decrease        = 1
   padapted mg sweeps pre      = 10
   padapted mg sweeps post     = 12
   padapted mg sweeps coarsest = 20
#end
\end{lstlisting}

\begin{longtable}{|p{4cm}|p{10cm}|p{2.2cm}|}
\caption{Keywords for the p-adaptation algorithms.} \label{tab:pAdaptationKey} \\
\hline
\multicolumn{1}{|c|}{\textbf{Keyword}} & \multicolumn{1}{c|}{\textbf{Description}} & \multicolumn{1}{c|}{\textbf{Default value}} \\ \hline
\endfirsthead

\caption{Keywords for the p-adaptation algorithms - continued.} \\
\hline
\multicolumn{1}{|c|}{\textbf{Keyword}} & \multicolumn{1}{c|}{\textbf{Description}} & \multicolumn{1}{c|}{\textbf{Default value}} \\ \hline
\endhead

truncation error type & \textit{CHARACTER}: Can be either "isolated" or "non-isolated". & isolated \\ \hline

truncation error & \textit{REAL}: Target truncation error for the p-adaptation algorithm. & \textbf{Mandatory keyword} \\ \hline

coarse truncation error & \textit{REAL}: Truncation error used for coarsening. & same as truncation error \\ \hline

Nmax          & \textit{INTEGER}(3): Maximum polynomial order in each direction for the p-adaptation algorithm. & 
					\textbf{Mandatory keyword} \\ \hline

Nmin          & \textit{INTEGER}(3): Minimum polynomial order in each direction for the p-adaptation algorithm. & 
					[1,1,1] \\ \hline

conforming boundaries & \textit{CHARACTER}(*): Specifies the boundaries of the geometry that must be forced to be conforming after the p-adaptation process.  	  & 
					-- \\ \hline

order across faces &
			\textit{CHARACTER}: Mathematical expression to specify the maximum polynomial order jump across faces. Currently, only $N*2/3$ and $N-1$ are supported. &
					$N-1$ \\ \hline 

increasing & \textit{LOGICAL}: If .TRUE. the multi-stage FMG adaptation algorithm is used. & 
					.FALSE. \\ \hline

write error files &
			\textit{LOGICAL}: If .TRUE., the program writes a file per element containing the directional tau-estimations. The files are stored in the folder \textit{./TauEstimation/}. When the simulation has several adaptation stages, the new information is just appended. & 
			 		.FALSE. \\ \hline

adjust nz & 
			\textit{LOGICAL}: If .TRUE., the order accross faces is adjusted i    n the directions xi, eta, and zeta of the face (being zeta the normal direction). If .FALSE., the     order is only adjusted in the xi and eta directions. The adjustment currently consists (hard-cod    ed) in allowing jumps in the polynomial order of at most 1. & 
					.FALSE. \\ \hline

mode & 
			\textit{CHARACTER}: p-Adaptation mode. Can be \textit{static}, \textit{time} or \textit{iteration}. Static p-adaptation is performed once at the beginning of a simulation for steady or unsteady simulations. Unsteady adaptation can be by \textit{time} or by \textit{iteration}. &
					\textit{static} \\ \hline

interval & 
			\textit{INTEGER/REAL}: In dynamic p-adaptation cases, this keyword specifies the iteration (integer) or time (real) interval for p-adaptation. &
					\textit{huge number} \\ \hline

restart files & 
			\textit{LOGICAL}: If .TRUE., the program writes restart files before and after the p-adaptation. &
					.FALSE. \\ \hline
					
max N decrease & 
			\textit{INTEGER}: Maximum decrease in the polynomial order in every p-adaptation procedure. &
					$N-N_{\textit{min}}$ \\ \hline

post smoothing residual & 
			\textit{REAL}: Specifies the maximum allowable deviation of $\partial_t q$ after the p-adaptation procedure. &
					-- \\ \hline
					
post smoothing method & 
			\textit{CHARACTER}: Either RK3 or FAS. &
					RK3, if the last keyword is activated \\ \hline

estimation files & 
			\textit{CHARACTER}: Name of the folder that contains the error estimations obtained with the multi tau-estimation (section \ref{sec:MultiTau}). &
					-- \\ \hline
					
estimation files number & 
			\textit{INTEGER(2)}: First and last estimation stages to be used for p-adaptation. &
					Mandatory if last keyword is used. \\ \hline

padapted $\ll \textit{keyword} \gg$ &
			\textit{MULTIPLE}: Specifies control file keywords that should be replaced after the adaptation procedure. Currently, only 'mg sweeps         ', 'mg sweeps pre', 'mg sweeps post', and 'mg sweeps coarsest' are supported. &
					-- \\ \hline

\end{longtable}

\section{Multiple truncation error estimations} \label{sec:MultiTau}
A static p-adaptation procedure can be driven by a set of error estimations, which have to be performed beforehand in a simulation with the following block:

\begin{lstlisting}
#define multi tau-estimation
   truncation error type = isolated
   interval              = 10
   folder                = MultiTau
#end
\end{lstlisting}


%%%%%%%%%%%%%%%%%%%%%%%%%%%%%%%%%%%%%%%%%%%%%%%%%%%%%%%%%%%%%%%%%%%%%%%%


\chapter{Monitors}

The monitors are specified individually as blocks in the control file.
The only general keyword that can be specified is explained in Table \ref{tab:monitorsKey}.



\begin{table}[h]
\caption{Keywords for monitors.} \label{tab:monitorsKey} 

\begin{tabular}{|p{4cm}|p{10cm}|p{2.2cm}|}
\hline
\multicolumn{1}{|c|}{\textbf{Keyword}} & \multicolumn{1}{c|}{\textbf{Description}} & \multicolumn{1}{c|}{\textbf{Default value}} \\ \hline

monitors flush interval 	& 
			\textit{INTEGER}: Iteration interval to flush the monitor information to the monitor files. & 
							100 \\ \hline

\end{tabular}
\end{table}

\section{Residual Monitors}

\section{Statistics Monitor}
\begin{lstlisting}
#define statistics
   initial time      = 1.d0
   initial iteration = 10
   sampling interval = 10
   dump interval     = 20
   @start
#end
\end{lstlisting}

By default, the statistic monitor will average following variables:

\begin{multicols}{3}
\begin{itemize}
\item u
\item v
\item w
\item uu 
\item vv 
\item ww 
\item uv 
\item uw 
\item vw 
\end{itemize}
\end{multicols}

A keyword preceded by @ is used in real-time to signalize the solver what it must do with the statistics computation:

\begin{multicols}{3}
\begin{itemize}
\item @start
\item @pause
\item @stop
\item @reset
\item @dump
\end{itemize}
\end{multicols}

After reading the keyword, the solver performs the desired action and marks it with a star, e.g. @start*.

\textbf{ATTENTION:} Real-time keywords may not work in parallel MPI computations. I depends on how the system is configured.

%%%%%%%%%%%%%%%%%%%%%%%%%%%%%%%%%%%%%%%%%%%%%%%%%%%%%%%%%%%%%%%%%%%%%%

\section{Probes}

\begin{lstlisting}
#define probe 1
   name     = SomeName
   variable = SomeVariable
   position = [0.d0, 0.d0, 0.d0]
#end
\end{lstlisting}

\begin{longtable}{|p{4cm}|p{10cm}|p{2.2cm}|}
\caption{Keywords for probes.} \label{tab:ProbesKey} \\
\hline
\multicolumn{1}{|c|}{\textbf{Keyword}} & \multicolumn{1}{c|}{\textbf{Description}} & \multicolumn{1}{c|}{\textbf{Default value}} \\ \hline
\endfirsthead

\caption{Keywords for the p-adaptation algorithms - continued.} \\
\hline
\multicolumn{1}{|c|}{\textbf{Keyword}} & \multicolumn{1}{c|}{\textbf{Description}} & \multicolumn{1}{c|}{\textbf{Default value}} \\ \hline
\endhead

name 	& 
			\textit{CHARACTER}: Name of the monitor. & 
							\textbf{Mandatory Keyword} \\ \hline

variable 	& 
			\textit{CHARACTER}: Variable to be monitored. Implemented options are:\
\begin{multicols}{3}
\begin{itemize}
\item pressure
\item velocity
\item u
\item v
\item w
\item mach
\item k
\end{itemize}			
\end{multicols}	
			 & 
							\textbf{Mandatory Keyword} \\ \hline

position 	& 
			\textit{REAL(3)}: Coordinates of the point to be monitored. & 
							\textbf{Mandatory Keyword} \\ \hline
\end{longtable}

%%%%%%%%%%%%%%%%%%%%%%%%%%%%%%%%%%%%%%%%%%%%%%%%%%%%%%%%%%%%%%%%%%%%%%%%

\section{Surface Monitors}


\begin{lstlisting}
#define surface monitor 1
   name              = SomeName
   marker            = NameOfBoundary
   variable          = SomeVariable
   reference surface = 1.d0
   direction         = [1.d0, 0.d0, 0.d0]
#end
\end{lstlisting}

\begin{longtable}{|p{4cm}|p{10cm}|p{2.2cm}|}
\caption{Keywords for probes.} \label{tab:SurfaceMonitorKey} \\
\hline
\multicolumn{1}{|c|}{\textbf{Keyword}} & \multicolumn{1}{c|}{\textbf{Description}} & \multicolumn{1}{c|}{\textbf{Default value}} \\ \hline
\endfirsthead

\caption{Keywords for the p-adaptation algorithms - continued.} \\
\hline
\multicolumn{1}{|c|}{\textbf{Keyword}} & \multicolumn{1}{c|}{\textbf{Description}} & \multicolumn{1}{c|}{\textbf{Default value}} \\ \hline
\endhead

name 	& 
			\textit{CHARACTER}: Name of the monitor. & 
							\textbf{Mandatory Keyword} \\ \hline
							
marker & 
			\textit{CHARACTER}: Name of the boundary where a variable will be monitored. & 
							\textbf{Mandatory Keyword} \\ \hline

variable 	& 
			\textit{CHARACTER}: Variable to be monitored. Implemented options are:\
\begin{multicols}{2}
\begin{itemize}
\item mass-flow
\item flow
\item pressure-force
\item viscous-force
\item force
\item lift
\item drag
\item pressure-average
\end{itemize}			
\end{multicols}	
			 & 
							\textbf{Mandatory Keyword} \\ \hline

reference surface 	& 
			\textit{REAL}: Reference surface [area] for the monitor. Needed for "lift" and "drag" computations. & 
							-- \\ \hline

direction 	& 
			\textit{REAL(3)}: Direction in which the force is going to be measured. Needed for "pressure-force", "viscous-force" and "force". Can be specified for "lift" (default [0.d0,1.d0,0.d0]) and "drag" (default [1.d0,0.d0,0.d0])   & 
							-- \\ \hline
\end{longtable}


\section{Volume Monitors}
Volume monitors compute the average of a quantity in the whole domain. They can be scalars(s) or vectors(v). 

\begin{lstlisting}
#define volume monitor 1
   name     = SomeName
   variable = SomeVariable
#end
\end{lstlisting}

\begin{longtable}{|p{4cm}|p{10cm}|p{2.2cm}|}
\caption{Keywords for volume monitors.} \label{tab:VolMonitorsKey} \\
\hline
\multicolumn{1}{|c|}{\textbf{Keyword}} & \multicolumn{1}{c|}{\textbf{Description}} & \multicolumn{1}{c|}{\textbf{Default value}} \\ \hline
\endfirsthead

\caption{Keywords for the p-adaptation algorithms - continued.} \\
\hline
\multicolumn{1}{|c|}{\textbf{Keyword}} & \multicolumn{1}{c|}{\textbf{Description}} & \multicolumn{1}{c|}{\textbf{Default value}} \\ \hline
\endhead

name 	& 
			\textit{CHARACTER}: Name of the monitor. & 
							\textbf{Mandatory Keyword} \\ \hline

variable 	& 
			\textit{CHARACTER}: Variable to be monitored. The variable can be scalar (s) or vectorial (v). Implemented options are:\
\begin{multicols}{2}
\begin{itemize}
\item[\textbf{(s)}] kinetic energy
\item[\textbf{(s)}] kinetic energy rate
\item[\textbf{(s)}] enstrophy
\item[\textbf{(s)}] entropy
\item[\textbf{(s)}] entropy rate
\item[\textbf{(s)}] mean velocity
\item[\textbf{(v)}] velocity
\item[\textbf{(v)}] momentum
\item[\textbf{(v)}] source
\end{itemize}		
\end{multicols}	
			
			 & 
							\textbf{Mandatory Keyword} \\ \hline


\end{longtable}

\chapter{Advanced User Setup}

Advanced users can have additional control over a simulation without having to modify the source code and recompile the code. To do that, the user can provide a set of routines that are called in different stages of the simulation via the Problem file (\textit{ProblemFile.f90}). A description of the routines of the Problem File can be found in section \ref{sec:ProblemFile}.

\section{Routines of the Problem File: \textit{ProblemFile.f90}} \label{sec:ProblemFile}

\begin{itemize}
\item UserDefinedStartup: Called before any other routines

\item UserDefinedFinalSetup: Called after the mesh is read in to allow mesh related initializations or memory allocations.

\item UserDefinedInitialCondition: called to set the initial condition for the flow. By default it sets an uniform initial condition, but the user can change it.

\item UserDefinedState1, UserDefinedNeumann: Used to define an user-defined boundary condition.

\item UserDefinedPeriodicOperation: Called before every time-step to allow periodic operations to be performed.

\item UserDefinedSourceTermNS: Called to apply source terms to the equation.

\item UserDefinedFinalize: Called after the solution computed to allow, for example error tests to be performed.

\item UserDefinedTermination: Called at the the end of the main driver after everything else is done.
\end{itemize}

\section{Compiling the Problem File}

The Problem Fie file must be compiled using a specific Makefile that links it with the libraries of the code. If you are using the \textit{horses/dev} environment module, you can get templates of the \textit{Problemfile.f90} and \textit{Makefile} with the following commands:

\begin{lstlisting}[language=bash]
	$ horses-get-makefile
	$ horses-get-problemfile
\end{lstlisting}

Otherwise, search the test cases for examples.\\

To run a simulation using user-defined operations, create a folder called SETUP on the path were the simulation is going to be run. Then, store the modified \textit{ProblemFile.f90} and the \textit{Makefile} in SETUP, and compile using:

\begin{lstlisting}[language=bash]
	$ make <<Options>>
\end{lstlisting}
where again the options are (bold are default):
\begin{itemize}
\item MODE=DEBUG/\textbf{RELEASE}
\item COMPILER=ifort/\textbf{gfortran}
\item COMM=PARALLEL/\textbf{SEQUENTIAL}
\item PLATFORM=MACOSX/\textbf{LINUX}
\item ENABLE\_THREADS=NO/\textbf{YES}
\end{itemize}

%%%%%%%%%%%%%%%%%%%%%%%%%%%%%%%%%%%%%%%%%%%%%%%%%%%%%%%%%%%%%%%%%%%%%%

\chapter{Postprocessing}

For postprocessing the Simulation Results

\section{Visualization with Tecplot Format: \textit{horses2plt}}

HORSES3D provides a script for converting the native binary solution files (*.hsol) into tecplot ASCII format (*.tec), which can be visualized in Pareview or Tecplot. Usage:

\begin{lstlisting}[language=bash]
	$ horses2plt SolutionFile.hsol MeshFile.hmesh <<Options>>
\end{lstlisting}

The options comprise following flags:

\begin{longtable}{|p{4cm}|p{10cm}|p{2.2cm}|}
\caption{Flags for \textit{horses2plt}.} \label{tab:Postprocessing} \\
\hline
\multicolumn{1}{|c|}{\textbf{Flag}} & \multicolumn{1}{c|}{\textbf{Description}} & \multicolumn{1}{c|}{\textbf{Default value}} \\ \hline
\endfirsthead

\caption{Additional flags for postprocessing with \textit{horses2plt}} \\
\hline
\multicolumn{1}{|c|}{\textbf{Flag}} & \multicolumn{1}{c|}{\textbf{Description}} & \multicolumn{1}{c|}{\textbf{Default value}} \\ \hline
\endhead

-{}-output-order= 	& 
			\textit{INTEGER}: Output order nodes. The solution is interpolated into the desired number of points. & 
							Not Present \\ \hline
							
-{}-output-basis= 	& 
			\textit{CHARACTER}: Either \textit{Homogeneous} (for equispaced nodes, or \textit{Gauss}.  & 
							\textit{Gauss}* \\ \hline
							
-{}-output-mode= 	& 
			\textit{CHARACTER}: Either \textit{multizone} or \textit{FE}. The option \textit{multizone} generates a Tecplot zone for each element. The option \textit{FE} generates only one Tecplot zone for the fluid and one for each boundary (if \textit{-{}-boundary-file} is defined).
			Each subcell is mapped as a linear finite element. This format is faster to read by Paraview and Tecplot.  & 
							\textit{multizone} \\ \hline
							
-{}-output-variables= 	& 
			\textit{CHARACTER}: Output variables separated by commas.A complete description can be found in Section \ref{PostProc:hsol}. & 
							Q \\ \hline

-{}-dimensionless 	& 
			Specifies that the output quantities must be dimensionless & 
							Not Present  \\ \hline
							
-{}-partition-file= 	& 
			\textit{CHARACTER}: Specifies the path to the partition file (*.pmesh) to export the MPI ranks of the simulation. & 
							Not Present  \\ \hline
							
-{}-boundary-file= 	& 
			\textit{CHARACTER}: Specifies the path to the boundary mesh file (*.bmesh) to export the surfaces as additional zones of the Tecplot file. & 
							Not Present  \\ \hline
\multicolumn{3}{p{16.4cm}}{*  \textit{Homogeneous} when \textit{-{}-output-order} is specified} \\

\end{longtable}

Additionally, depending on the type of solution file, the user can specify additional options.

\subsection{Solution Files (*.hsol)} \label{PostProc:hsol}

For standard solution files, the user can specify which variables they want to be exported to the Tecplot file with the flag \textit{-{}-output-variables=}. 
The options are:

\begin{multicols}{5}
\begin{itemize}
\item $Q$ (default)
\item $rho$
\item $u$
\item $v$
\item $w$
\item $p$
\item $T$
\item $Mach$
\item $s$
\item $Vabs$
\item $V$
\item $Ht$
\item $rhou$
\item $rhov$
\item $rhow$
\item $rhoe$
\item $c$
\item $Nxi$
\item $Neta$
\item $Nzeta$
\item $Nav$
\item $N$
\item $Ax\_Xi$
\item $Ax\_Eta$
\item $Ax\_Zeta$
\item $ThreeAxes$
\item $Axes$
\item $mpi\_rank$
\item $eID$
\item $gradV$
\item $u\_x$
\item $v\_x$
\item $w\_x$
\item $u\_y$
\item $v\_y$
\item $w\_y$
\item $u\_z$
\item $v\_z$
\item $w\_z$
\item $c\_x$
\item $c\_y$
\item $c\_z$
\item $omega$
\item $omega\_x$
\item $omega\_y$
\item $omega\_z$
\item $omega\_abs$
\item $Qcrit$
\end{itemize}
\end{multicols}

\subsection{Statistics Files (*.stats.hsol)}
Statistics files generate following variables by default (being Sij the components of the Reynolds Stress tensor):

\begin{multicols}{3}
\begin{itemize}
\item Umean
\item Vmean
\item Wmean
\item Sxx
\item Syy
\item Szz
\item Sxy
\item Sxz
\item Syz
\end{itemize}
\end{multicols}

\section{Extract geometry}
Under construction.

\section{Merge statistics tool}

Tool to merge several statistics files. The usage is the following:

\begin{lstlisting}[language=bash]
	$ horses.mergeStats *.hsol --initial-iteration=INTEGER --file-name=CHARACTER
\end{lstlisting}

Some remarks:
\begin{itemize}
\item Only usable with statistics files that are obtained with the "reset interval" keyword and/or with individual consecutive simulations. 
\item Only constant time-stepping is supported.
\item Dynamic p-adaptation is currently not supported.
\end{itemize}


\bibliography{../LaTeX/9_backmatter/library}

\end{document}
