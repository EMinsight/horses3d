\documentclass[a4paper,10pt]{report}
%\usepackage[spanish]{babel}
\usepackage[utf8]{inputenc}
\usepackage{graphicx}
\usepackage{float} 
\usepackage{multirow}
\usepackage{vmargin}
\usepackage{amsmath}
\usepackage{amsfonts}
\usepackage{subfigure}
\usepackage{afterpage}
\usepackage[calcwidth]{titlesec}
\usepackage{verbatim}
\usepackage[hidelinks]{hyperref}
\usepackage{multicol}
\usepackage{pfnote}
\usepackage{fnpos}
\usepackage{color}
\usepackage{xcolor}
\usepackage{listings}
\usepackage{algorithm,algpseudocode}
\usepackage{epstopdf}
\usepackage{tcolorbox}
\setmarginsrb{2cm}{2cm}{2cm}{2cm}{0cm}{0cm}{0cm}{0.5cm}%{left}{top}{right}{bottom}{headhgt}{}
%\numberwithin{equation}{section}
%Bibliography style:
\bibliographystyle{elsarticle-num}
%\biboptions{sort&compress}
%opening

\definecolor{mygreen}{RGB}{28,172,0} % color values Red, Green, Blue
\definecolor{mylilas}{RGB}{170,55,241}


\title{\textbf{NSLITE3D \\ User Manual}}
\author{Andrés Rueda \\ Many others (future?)}

\begin{document}

\lstset{backgroundcolor = \color{lightgray},
    language=Matlab,%
    %basicstyle=\color{red},
    breaklines=true,%
    morekeywords={matlab2tikz},
    keywordstyle=\color{blue},%
    morekeywords=[2]{1}, keywordstyle=[2]{\color{black}},
    identifierstyle=\color{black},%
    stringstyle=\color{mylilas},
    commentstyle=\color{mygreen},%
    showstringspaces=false,%without this there will be a symbol in the places where there is a space
    numbers=left,%
    numberstyle={\tiny \color{black}},% size of the numbers
    numbersep=9pt, % this defines how far the numbers are from the text
    emph=[1]{for,end,break},emphstyle=[1]\color{red}, %some words to emphasise
    %emph=[2]{word1,word2}, emphstyle=[2]{style},    
}

\maketitle

\tableofcontents

\chapter{Implicit solvers}
\section{Keywords}
The keywords for the implicit solvers are listed in table \ref{tab:implicitkey}

\begin{table}[htbp]
\caption{Keywords for implicit solvers.}
\begin{tabular}{|l|p{10cm}|l|}
\hline
\multicolumn{1}{|c|}{Keyword} & \multicolumn{1}{c|}{Description} & \multicolumn{1}{c|}{Default value} \\ \hline
implicit time & \textit{LOGICAL}: When .TRUE., NSLITE3D performs implicit time integration in every time step. & \multicolumn{1}{c|}{.FALSE.} \\ \hline
jacobian flag           & \textit{INTEGER}: Specifies the type of Jacobian matrix to be computed. Options are:\
				\begin{enumerate}
					\item Jacobian free: Uses JFNK algorithm with GMRES linear solver.
					\item Numerical Jacobian: Uses coloring algorithm for computing Jacobian.
					\item Analytical Jacobian: Not yet implemented.
					\end{enumerate}
										& \multicolumn{1}{c|}{1} \\ \hline
time integration        & \textit{CHARACTER}: Specifies if NSLITE3D must perform a 'steady-state' or a 'time-accurate' simulation. &  'steady-state'\\ \hline
jacobian by convergence & \textit{LOGICAL}: When .TRUE., the Jacobian is only computed when the convergence falls beneath some threshold (see keyfords: blah and blah blah). This improves performance but can introduce big numerical errors for time-accurate simulations.  & .FALSE. \\ \hline
linear solver           & \textit{CHARACTER}: Specifies the linear solver that has to be used. Options are:\ 
				\begin{itemize}
					\item 'petsc': PETSc library Krylov-Subspace methods.
					\item 'pardiso': Intel MKL PARDISO.
				\end{itemize}
										& 'petsc'  \\ \hline
  &  &  \\ \hline
   &  &  \\ \hline
    &  &  \\ \hline
\end{tabular}
\label{tab:implicitkey}
\end{table}


\chapter{Restarting a case}

\begin{table}[htbp]
\caption{Keywords for restarting a case.}
\begin{tabular}{|l|p{10cm}|p{2cm}|}
\hline
\multicolumn{1}{|c|}{Keyword} & \multicolumn{1}{c|}{Description} & \multicolumn{1}{c|}{Default value} \\ \hline
restart file name   & \textit{CHARACTER}: Name of the restart file to be written and, if keyword \textit{restart} = .TRUE., also name of the restart file to be read for starting the simulation. & \textbf{Mandatory\ keyword} \\ \hline
restart 			& \textit{LOGICAL}: If .TRUE., initial conditions of simulation will be read from restart file specified using the keyword \textit{restart file name}. & \textbf{Mandatory keyword} \\ \hline
restart interval    & \textit{INTEGER}: Indicates how often restart files have to be written. & Huge number \\ \hline
\end{tabular}
\label{tab:restartkey}
\end{table}





\bibliography{../LaTeX/9_backmatter/library}

\end{document}

